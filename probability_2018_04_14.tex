\section{Лекция от 14.04.2018}
\begin{theorem}[Колмогорова-Хинчина о сходимости ряда]
	Пусть $\{\xi_n \}_{n \geqslant 1}$~--- последовательность независимых случайных величин такая, что $\E \xi_n = $ и $\E \xi^2 < +\infty$. Тогда, если $\sum\limits_{n = 1}^{\infty} \E \xi_n^2 < +\infty$, то $ \sum\limits_{n = 1}^{\infty} \xi_n$ сходится почти наверное.
	\begin{proof}
		Обозначим $S_n = \sum\limits_{k=1}^n \xi_k$. По критерию Коши $\left\{ \sum\limits_{n = 1}^{\infty}~\text{сходится п.н.} \right\}$ равносильно тому, что $\{ S_n~\text{фундаментально п.н.}\}$, а это в свою по критерию фундаментальности равносильно тому, что 
		$$\forall \varepsilon > 0: \P \left( \sup\limits_{k \geqslant n} | S_k - S_n | \geqslant \varepsilon \right) \limn 0.$$
		 Очевидно,
		 $$\P \left( \sup\limits_{k \geqslant n} | S_k - S_n | \geqslant \varepsilon \right) = \P \left( \bigcup\limits_{k \geqslant n} \big\{| S_k - S_n | \geqslant \varepsilon \big\} \right),$$
		 а из непрерывности вероятностной меры следует, что 
		 $$\lim\limits_{N \rightarrow +\infty} \P \left( \bigcup\limits_{k = n}^{N} \big\{| S_k - S_n | \geqslant \varepsilon \big\} \right) = \lim\limits_{N \rightarrow +\infty} \P \left( \max\limits_{n \leqslant k \leqslant N} | S_k - S_n | \geqslant \varepsilon \right).$$
		 По неравенство Колмогорова это меньше или равно, чем
		 $$\lim\limits_{N \rightarrow + \infty} \dfrac{\E ( S_N - S_n)^2}{\varepsilon^2} = \lim\limits_{N \rightarrow + \infty} \dfrac{1}{\varepsilon^2} \sum\limits_{k = n + 1}^N \E \xi_k^2 = \dfrac{1}{\varepsilon^2} \sum\limits_{k > n} \E \xi_k^2 \limn 0.$$
	\end{proof}
\end{theorem}
\begin{lemma}[Тёплица]
	Пусть $x_n \rightarrow x$~--- числовая последовательность, числа $\{a_n\}_{n \geqslant 1}$ таковы, что $ \forall n: a_n \geqslant 0$ и $ b_n = \sum\limits_{k = 1}^n a_k \uparrow +\infty$. Тогда $\dfrac{1}{b_n} \sum\limits_{i = 1}^n a_i x_i \limn 0$.
	\begin{proof}
		Пусть $\varepsilon > 0$. Выберем $n_0$ так, что $\forall n > n_0: |x_n - x| \leqslant \dfrac{\varepsilon}{2}$. Выберем $n_1 > n_0$ такое, что $\dfrac{1}{b_n} \sum\limits_{k=1}^{n_0} a_k | x_k - x| \leqslant \frac{\varepsilon}{2}$, тогда
		\begin{multline*}
			\forall n > n_1: \left| \dfrac{1}{b_n} \sum\limits_{k = 1}^n a_k x_k - x \right| = \left| \dfrac{1}{b_n} \sum\limits_{k = 1}^n a_k x_k - \dfrac{1}{b_n} \sum\limits_{k = 1}^n a_k x \right| \leqslant \dfrac{1}{b_n} \sum\limits_{k = 1}^n a_k |x_k - x| = \\= \dfrac{1}{b_n} \sum\limits_{k = 1}^{n_0} a_k |x_k - x| + \dfrac{1}{b_n} \sum\limits_{k = n_0 + 1}^n a_k |x_k - x| \leqslant \dfrac{\varepsilon}{2} + \dfrac{\varepsilon}{2} \cdot \dfrac{1}{b_n} \sum\limits_{k = n_0 + 1}^n a_k \leqslant \varepsilon.
		\end{multline*}
	\end{proof}
\end{lemma}
\begin{lemma}[Кронекера]
	Пусть ряд $\sum\limits_{n = 1}^{\infty} x_n$ сходится, $\{a_n\}_{n \geqslant 1}$ такова, что $a_n \geqslant 0$, $b_n = \sum\limits_{k=1}^n a_k \uparrow + \infty$. Тогда $\dfrac{1}{b_n} \sum\limits_{k = 1}^n a_k x_k \limn 0$.
	\begin{proof}
		Пусть $S_n = \sum\limits_{k = 1}^n x_k$, тогда $S_n \limn S = \sum\limits_{k = 1}^\infty x_k$. Заметим,
		\begin{multline*}
			\sum\limits_{j = 1}^n b_j x_j = \sum\limits_{j = 1}^n b_j (S_j - S_{j - 1}) = b_n S_n - \sum\limits_{j = 1}^n S_{j-1}(b_j - b_{j-1}) = b_n S_n - \sum\limits_{j = 1}^n S_{j-1} a_j.
		\end{multline*}
		Следовательно,
		$$\dfrac{1}{b_n} \sum\limits_{k = 1}^n a_k x_k = S - \cancelto{\scriptsize S~\text{по Тёплицу}}{\dfrac{1}{b_n}\sum\limits_{j = 1}^n S_{j-1} a_j} \limn 0.$$
	\end{proof}
\end{lemma}
\begin{theorem}[усиленный закон больших чисел в форме Колмогорова-Хинчина]
	Пусть $\{\xi_n\}_{n \geqslant 1}$~--- независимые случайные величины, $\forall n: \D \xi_n < +\infty$. Пусть $\{ b_n \}_{n \geqslant 1}$~--- числовая последовательность, $b_1 > 0$ и $b_n \uparrow +\infty$, причем $\sum\limits_{n =1}^\infty \dfrac{\D \xi_n}{b_n^2} < + \infty$. Пусть $ S_n = \sum\limits_{i = 1}^{n} \xi_i$, тогда $\dfrac{S_n - \E S_n}{b_n} \xrightarrow[n \rightarrow + \infty]{\text{п.н.}} 0$.
	\begin{proof}
		Преобразуем:
		$$ \frac{S_n - \E S_n}{b_n} = \frac{1}{b_n} \sum\limits_{i = 1}^{n} b_i \cdot \frac{\xi_i - \E \xi_i}{b_i}.$$
		Обозначим $\eta_i = \dfrac{\xi_i - \E \xi_i}{b_i}$. Случайные величины $\eta_i$ независимы и  $\E \eta_i = 0$. Значит,  
		$$ \sum\limits_{i =1}^\infty \E \eta_i^2 = \sum\limits_{i = 1}^\infty \dfrac{\E( \xi_i - \E \xi_i)^2}{b_i^2} = \sum\limits_{i = 1}^\infty \dfrac{\D \xi_i}{b_i^2} < +\infty.$$
		Следовательно, по теореме Колмогорова-Хинчина о сходимости ряда $\sum \eta_i$ сходится почти наверное. По лемме Кронекера последовательность 
		$$\dfrac{1}{b_n} \sum\limits_{i =1}^n b_i \cdot \dfrac{\xi_i - \E \xi_i}{b_i}$$ 
		сходится к нулю для всех $\omega$, для которых сходится ряд 
		$$\sum\limits_{i = 1}^\infty \dfrac{\xi_i - \E \xi_i}{b_i} = \sum\limits_{i = 1}^\infty \eta_i $$ 
		сходится. Следовательно, 
		$$ \dfrac{1}{b_n} \sum\limits_{i =1}^n b_i \cdot \dfrac{\xi_i - \E \xi_i}{b_i} = \frac{S_n - \E S_n}{b_n} \xrightarrow[n \rightarrow \infty]{\text{п.н}} 0.$$
	\end{proof}
\end{theorem}
\begin{lemma}
	Пусть 
\end{lemma}